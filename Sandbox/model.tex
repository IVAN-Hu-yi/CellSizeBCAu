\documentclass[12pt]{article}
\usepackage{graphicx} % Required for inserting images
\usepackage{url}
\usepackage{hyperref}
\usepackage{caption}
\usepackage{amsmath}
\usepackage{biblatex}

\title{The Role of Cell Size on bacterial community assembly}
\author{Yingcai Hu}


\begin{document}

\maketitle

\section{Introduction}



\section{Methods}

\subsection{Defining bacterial taxa and resource types}

In this study, each bacterial taxon $i$ are defined by their resource preference and average cell mass $m_i$. The preferred resources for each taxa are chosen uniformly

\subsection{Microbial Consumer-Resource Model (MiCRM)}

Here we adopt the framework from @iiiAvailableEnergyFluxes2019: the energy per unit time or, The metabolic rate, $v^{in}_{ij}$ of species $i$ using resource $j$ depends on the concentration of resource $j$, $R_j$ and species preference on a particular resource type $p_{ij}$:

\begin{equation}
    v^{in}_{ij} =  \sigma(p_{ij} R_j) = \frac{v_{max}p_{ij}R_{j}}{R_h + R_{j}}
\end{equation}


where the function $\sigma(x) = x_{\max} \frac{x}{k + x}$ is the Monod function that maps the resource availabe to the resources taken by bacteria. The total mass, $ v^{grow}_i $, used for growth of biomass will be defined by setting a fraction ($l_{j}$) of mass returned to the environment:

\begin{equation}
    v^{grow}_{i} = \sum^M_{j=1} (1-l_{j}) v^{in}_{ij}
\end{equation}

Therefore, assuming there are $N$ species and $M$ types of resouces,
the dynamics of biomass abundance $C_i$ of species $i^{th}$ and resource concentration $R_j$ of type $j^{th}$  is

\begin{equation} 
    \frac{dC_i}{dt} = \mu C_i  ( v^{grow}_i  - \phi_i )
\end{equation}

 \begin{equation}
    \frac{dR_j}{dt} = \rho_j - k_{m} \sum^N_{i=1} C_i (v^{out}_{ij} - v^{in}_{ij})
 \end{equation}

 Here, $U_{ij}$ is uptake rate of resource $j$ by speceis $i$, $l_{ij}$ is the leakage of resource $j$ to the environment in the form of resource $k$. $m_i$ is the maintainence required for species $i$. The terms are explained in Table 1. 

 \begin{center}
    \captionof{table}{Definition and Units of Parametres}
    \begin{tabular}{ |c|c|c| } 
     \hline
     Symbols & Definition & Units \\

     \hline 
     $C$ & Biomass content & $mass$ \\ 
     $R$ & Resouces content & $mass$ \\
     $U$ & Uptake rate & $time^{-1}$ \\
     $l$ & Fraction of leakage & $None$ \\
     $m$ & Maintainence coefficient & $time^{-1}$ \\
     $\rho$ & External resource supply & $mass/time$ \\
     $\mu$ &  Constant that scales the resource uptake to growth & $mass^{-1}$ \\
     $k_m$ & Constant that scales the maintanence & $mass^{-1}$ \\
     $k_{ab}$ & Constant that scales the body mass and resource intake & $mass^{-1}$ \\
     \hline

    \end{tabular}
    \end{center}

Here $v_{ij}$, $v^{grow}$ and $\phi_i$ are scaled by biomass for investigating the effect of biomass on population dynamics within the bacterial community. 

\subsection{Body mass Parametrisation on MiCRM}

\subsection{Linear Stability Analysis}

Let $x=[C_1, C_2, \dots, C_N, R_1, R_2, \dots, R_M] \in S $, then $x^* \in S $ is the steady state (i.e. $f'_n(x^*) = 0$ and $f'_{N+m}(x^*)=0$ for $n=1, 2, 3,\dots, N, m=1, 2, 3, \dots, N$) . The jocobian matrix of the MiRCM will be defined as followed:
\[ J = 
\begin{bmatrix}

    \frac{\partial{f_1}}{\partial{C_1}} & \frac{\partial{f_1}}{\partial{C_2}} & \dots & \frac{\partial{f_1}}{\partial{C_N}} & \frac{\partial{f_1}}{\partial{R_1}} & \frac{\partial{f_1}}{\partial{R_2}} & \dots & \frac{\partial{f_1}}{\partial{R_M}}\\
    \frac{\partial{f_2}}{\partial{C_1}} & \frac{\partial{f_2}}{\partial{C_2}} & \dots & \frac{\partial{f_2}}{\partial{C_N}} & \frac{\partial{f_2}}{\partial{R_1}} & \frac{\partial{f_2}}{\partial{R_2}} & \dots & \frac{\partial{f_2}}{\partial{R_M}}\\
    \vdots & \vdots  & \ddots & \vdots & \vdots & \vdots & \ddots & \vdots\\
    \frac{\partial{f_N}}{\partial{C_1}} & \frac{\partial{f_N}}{\partial{C_2}} & \dots & \frac{\partial{f_N}}{\partial{C_N}} & \frac{\partial{f_N}}{\partial{R_1}} & \frac{\partial{f_N}}{\partial{R_2}} & \dots & \frac{\partial{f_N}}{\partial{R_M}}\\
    \frac{\partial{f_{N+1}}}{\partial{C_1}} & \frac{\partial{f_{N+1}}}{\partial{C_2}} & \dots & \frac{\partial{f_{N+1}}}{\partial{C_N}} & \frac{\partial{f_{N+1}}}{\partial{R_1}} & \frac{\partial{f_{N+1}}}{\partial{R_2}} & \dots & \frac{\partial{f_{N+1}}}{\partial{R_M}}\\
    \frac{\partial{f_{N+2}}}{\partial{C_1}} & \frac{\partial{f_{N+2}}}{\partial{C_2}} & \dots & \frac{\partial{f_{N+2}}}{\partial{C_N}} & \frac{\partial{f_{N+2}}}{\partial{R_1}} & \frac{\partial{f_{N+2}}}{\partial{R_2}} & \dots & \frac{\partial{f_{N+2}}}{\partial{R_M}}\\
    \vdots & \vdots  & \ddots & \vdots & \vdots & \vdots & \ddots & \vdots\\
    \frac{\partial{f_{N+M}}}{\partial{C_1}} & \frac{\partial{f_{N+M}}}{\partial{C_2}} & \dots & \frac{\partial{f_{N+M}}}{\partial{C_N}} & \frac{\partial{f_{N+M}}}{\partial{R_1}} & \frac{\partial{f_{N+M}}}{\partial{R_2}} & \dots & \frac{\partial{f_{N+M}}}{\partial{R_M}}\\
    
    \end{bmatrix}
    \]

The $f_n$ and $f_{N+m}$ are vector-valued function that returns Equation 3 and Equation 4, therefore the elements in $J$ are defined as:
\begin{equation}
    \frac{\partial{f_n}}{\partial{C_n}} = \mu (v^{grow}_n - \phi_n)
\end{equation}

\begin{equation}
    \frac{\partial{f_n}}{\partial{R_m}} = \mu C_n (1-l_m) \frac{v^{max}_{nm} p_{nm} R_m}{(R_h + p_{nm}R_m)^2}
\end{equation}

\begin{equation}
    \frac{\partial{f_{N+m}}}{\partial{C_n}} = k_m C_n (v^{out}_{nm} - v^{in}_{nm})
\end{equation}

\begin{equation}
    \frac{\partial{f_{N+m}}}{\partial{R_m}} = k_m \displaystyle \sum^N_{i=1} C_i (D_{mm}l_m - 1) \frac{v^{max}_{nm} p_{nm} R_m}{(R_h + p_{nm}R_m)^2}
\end{equation}

To investigate the stability of the system on steady state $x^*$, eigenvalue decomposition will be performed on the jocobian evaluated at $x^*$, i.e., $J_{|x=x^*} = P^{-1} \Lambda P $, using numpy library.

\end{document}
